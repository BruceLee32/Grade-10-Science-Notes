\documentclass[working]{tuftebook}

\usepackage{standalone}
\input{../preamble/preamble.tex}
\input{../preamble/laterPreamble.tex}

\usepackage{lewis}
\usepackage{bohr}
\usepackage[version=4]{mhchem}
\pagenumbering{gobble}

\declaretheorem[numbered=no, style=thmgreenbox, name=Word Equation]{wordequation}
\declaretheorem[numbered=no, style=thmgreenbox, name=Skeleton Equation]{skeletonequation}
\declaretheorem[numbered=no, style=thmgreenbox, name=Balanced Chemical Equation]{balancedchemicalequation}
\declaretheorem[numbered=no, style=thmgreenbox, name=WHMIS Safety Symbols]{whmis}

\begin{document}
\chapter*{Table Salt}
\vspace{-2em}
Table salt, also know as \emph{sodium chloride}, are crystals that are transparent, colourless, and brittle.\sidecite[]{saltassociation}

A visual representation of salt is seen in Figure \ref{fig:salt}. 


\begin{marginfigure}
    \centering 
    \includegraphics[width=0.5\textwidth]{../figures/table salt.jpg}
    \caption{An image of salt}
    \label{fig:salt}
\end{marginfigure}

\begin{wordequation}
Salt is made from a \textbf{synthesis reaction}, where sodium and chlorine are the reactants and sodium chloride is the product. 
    \[
        \text{sodium}+ \text{chlorine}\to \text{sodium chloride}
    \]
\end{wordequation}

\begin{skeletonequation}
    The skeleton equation for table salt is 
    \[
        \ce{Na_{(s)}}+ \ce{Cl_{2(g)}}\to \ce{NaCl_{(s)}}
    \]
    Where $\ce{Na_{(s)}}$ is sodium and $\ce{Cl_{2(g)}}$ is chlorine (diatomic molecule).
\end{skeletonequation}

\begin{balancedchemicalequation}
    The balanced chemical equation is 
    \[
        \ce{2Na_{(s)}}+ \ce{Cl_{2(g)}}\to \ce{2NaCl_{(s)}}
    \]
\end{balancedchemicalequation}

\begin{marginfigure}
    \centering
    \includegraphics[width=\textwidth]{../figures/road salt.jpg}
    \caption{Road salts are used to melt ice}
    \label{road-salt}
\end{marginfigure}

\textbf{Common Uses}\\
Some common uses for salt are 
\begin{enumerate}
    \item{Cooking; salt is most known for being a food preservative and used as a flavoring agent. It is also used for seasoning foods such as steak, soup, and chicken.}
    \item{Road salt; is used to melt ice to prevent cars and people from slipping (see Figure \ref{road-salt}). The salt compound decomposes into sodium and chlorine ions which disrupt bonds between water molecules.}
    \item{Removing stains; salt has powerful dehydration properties that will eliminate stains such as blood stains.}
\end{enumerate}

\textbf{Environmental Impacts}
\begin{enumerate}
    \item{The chemical reaction that makes table salt doesn't cause emission of any green house gases whatsoever.}
    \item{Table salt can contaminate water by making it too salty, and thus undrinkable. This can harm marine animals and wildlife.}
    \item{Table salt is generally environmentally friendly, unless used in high concentrations. For instance, road salt is damaging to aquatic animals.}
\end{enumerate}

\section*{Reactant: Chlorine}
Chlorine is a yellow-green gas at room temperature that is poisonous at high concentrations.\sidecite[]{chlorine}

\begin{marginfigure}
    \centering
    \vspace{2em}
    \includegraphics[width=0.5\textwidth]{../figures/chlorine_gas.png}
    \caption{Chlorine gas in an erlenmeyer flask}
    \label{fig:chlorine}
\end{marginfigure}

\begin{whmis}
    The WHMIS safety symbols are 

    \centering
    \includegraphics[width=0.7\textwidth]{../figures/whmis.png}
    
\end{whmis}

\textbf{Respective Symbols:} Oxidizing hazard, gas under pressure, acute toxicity, corrosive, can cause damage to environment. \sidecite[]{chlorine-whmis}
\vspace{2em}

\textbf{Chlorine: 3 Safety Tips}\sidecite[]{chlorine-whmis}
\begin{enumerate}
    \item{Do not get in eyes or on skin.}
    \item{Contain gas under pressure.}
    \item{Maintain suitable ventilation.}
\end{enumerate}
\vspace{2em}

\textbf{Chlorine: 2 Pieces of PPE}\sidecite[]{chlorine-whmis}
\begin{enumerate}
    \item{Put on appropriate personal protective equipment, such as a hazmat suit (see Figure \ref{fig:hazmat-suit}).}
    \item{Equip a respiratory when ventilation is poor.}
\end{enumerate}

\begin{marginfigure}
    \centering
    \includegraphics[width=0.5\textwidth]{../figures/hazmat suit.png}
    \caption{Hazmat suit}
    \label{fig:hazmat-suit}
\end{marginfigure}
\vspace{2em}

\textbf{Chlorine: 2 Tips for When it Goes Wrong}\sidecite[]{what-to-do}
\begin{enumerate}
   \item{Escape the area infected by the chlorine as soon as possible and breathe fresh air.}
   \item{Seek higher ground\sidenote{You want to seek higher ground because chlorine is heavier than air, and will thus sink.}, wash your entire body while holding your breathe, and seek help immediately.}
\end{enumerate}

\end{document}
