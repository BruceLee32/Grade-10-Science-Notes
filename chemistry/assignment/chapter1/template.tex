\documentclass[working]{tuftebook}

\usepackage{standalone}
%%%%%%%%%%%%%%%%%%%%%%%%%%
%%%%%% UNIT PREAMBLE %%%%% 
%%%%%%%%%%%%%%%%%%%%%%%%%%

% Basics 
\usepackage[utf8]{inputenc}
\usepackage[T1]{fontenc}
\usepackage{textcomp}
\usepackage[a4paper, left=1in, right=1in, top=1in, bottom=1in]{geometry}
\renewcommand\familydefault{\sfdefault}

\usepackage[Sonny]{fncychap}

\usepackage{amsmath,amsfonts,amsthm,amssymb,mathtools}
\usepackage[varbb]{newpxmath}
\usepackage{xfrac}
\usepackage[usenames,dvipsnames]{xcolor} % usenames, dvipsnames adds more colours
\usepackage{hhline}
\usepackage{comment}
\usepackage{tasks}
\usepackage{enumerate} 
\usepackage{enumitem} 
\usepackage{titlesec}
\usepackage[most]{tcolorbox}
\usepackage{lipsum}
\usepackage{tabularx}
\usepackage[labelfont=bf]{caption}
\usepackage{subfig}
\usepackage{pgfplots}
\usepackage{cancel} 
\usepackage{physics} 
\usepackage[bookmarks]{hyperref}
\usepackage{array}
\usepackage{float}
\usepackage{standalone}
\usepackage{graphicx}
\usepackage{forest}

% Tables 
\numberwithin{table}{section}

% Inkscape Figures
\usepackage{import}
\usepackage{xifthen}
\usepackage{pdfpages}
\usepackage{transparent}
\newcommand{\incfig}[2][1]{
    \def\svgwidth{#1\columnwidth}
    \import{../figures/}{#2.pdf_tex}
}

\pdfsuppresswarningpagegroup=1

% Chemistry
\usepackage{lewis} 
\usepackage{bohr}
\usepackage[version=4]{mhchem}

% Page setup
\hypersetup{hidelinks}
\pagenumbering{arabic}
\pagestyle{plain}
\setlength{\parindent}{0pt}

% Show subsubsections
\setcounter{tocdepth}{3}
\setcounter{secnumdepth}{3}

% Required for the Grid
\usetikzlibrary{calc}

% Section Font-Size
\titleformat{\subsubsection}
  {\normalfont\fontsize{12}{12}\bfseries}{\thesubsubsection}{1em}{}

\titleformat{\subsection}
  {\normalfont\fontsize{14}{14}\bfseries}{\thesubsection}{1em}{}

\titleformat{\section}
  {\normalfont\fontsize{16}{16}\bfseries}{\thesection}{1em}{}

% New page for each section 
\newcommand{\sectionbreak}{\clearpage}

% Section Spacing
\titlespacing{\section}{0em}{2.5em}{1em}
\titlespacing{\subsection}{0em}{2.5em}{1em}
\titlespacing{\subsubsection}{0em}{2.5em}{1em}

% TABLE COLUMN SEPARATION (USES ARRAY PACKAGE)
% \renewcommand{\arraystretch}{1.8} % changes vertical space for each cell 
% \setlength{\tabcolsep}{18pt} % changes horizontal space for each cell
% \setlength{\arrayrulewidth}{0.25mm}

% TCOLORBOX 
% \newtcolorbox[auto counter, number within=section]{definition}{colback=white,title=Example~\thetcbcounter,breakable,colframe=white,boxrule=0pt, enhanced, title style={left color=gray!60,right color=white,middle color=white},arc=0mm, titlerule=0pt, fonttitle=\bfseries\sffamily}

% Theorems 
\usepackage{thmtools}
\usepackage[framemethod=TikZ]{mdframed}

\declaretheoremstyle[
    headfont=\bfseries\sffamily\color{ProcessBlue!70!black}, bodyfont=\normalfont,
    headpunct= :,
    mdframed={
        linewidth=2pt,
        rightline=false, topline=false, bottomline=false,
        linecolor=ProcessBlue, backgroundcolor=ProcessBlue!5,
        innerbottommargin=10pt
    } ]{note}

\declaretheoremstyle[
    headfont=\bfseries\sffamily\color{NavyBlue!70!black}, 
    bodyfont=\normalfont,
    % headpunct=,
    mdframed={
        linewidth=2pt,
        rightline=false, topline=false, bottomline=false, linecolor=NavyBlue, innerbottommargin=10pt
    }
]{solution}

\declaretheoremstyle[
    headfont=\bfseries\sffamily\color{Gray!70!black}, bodyfont=\normalfont,
    % headpunct= ,
    postheadspace=\newline,
    mdframed={
        linewidth=2pt,
        rightline=false, topline=false, bottomline=false,
        linecolor=Gray, backgroundcolor=Gray!5,
        innerbottommargin=10pt
    } ]{remark}

\declaretheoremstyle[
    headfont=\bfseries\sffamily\color{Fuchsia!70!black}, bodyfont=\normalfont,
    % headpunct= ,
    mdframed={
        linewidth=2pt,
        rightline=false, topline=false, bottomline=false,
        linecolor=Fuchsia, backgroundcolor=Fuchsia!5,
        innerbottommargin=10pt
    }
]{example}

\declaretheoremstyle[
    headfont=\bfseries\sffamily\color{Fuchsia!70!black}, 
    bodyfont=\normalfont,
    % headpunct= ,
    mdframed={
        linewidth=2pt,
        rightline=false, topline=false, bottomline=false,
        linecolor=Fuchsia,
    }
]{examplesolution}

\declaretheoremstyle[
    headfont=\bfseries\sffamily\color{black!70!black}, 
    bodyfont=\normalfont,
    mdframed={
        linewidth=1pt,
        rightline=false, topline=false, bottomline=false,
        linecolor=black,
    }
]{definition}

\declaretheorem[style=solution, name=Solution, numbered=no]{solution}

\declaretheorem[style=solution, name=Derivation, numbered=no]{derivation}

\declaretheorem[style=definition, name=Definition, numberwithin=chapter]{definition}

\declaretheorem[style=note, name=Note, numbered=no]{noteswap}
\newenvironment{note}[1]{\vspace{0.5em}\begin{noteswap}[#1]}{\end{noteswap}\vspace{0.5em}}

\declaretheorem[style=remark, name=Remark, numbered=no]{remarkswap}
\newenvironment{remark}{\vspace{0.5em}\begin{remarkswap}}{\end{remarkswap}\vspace{0.5em}}

\declaretheorem[style=example, name=Example, numbered=no]{exampleswap}
% \newenvironment{example}{\vspace{0.5em}\begin{exampleswap}}{\end{exampleswap}}

\declaretheorem[style=examplesolution, name=Solution, numbered=no]{examplesolutionswap}
\newenvironment{examplesolution}{\vspace{-2em}\begin{examplesolutionswap}}{\end{examplesolutionswap}}

% Enumerate environments 
\newenvironment{2qu}
{
\begin{enumerate}[label=(\alph*)]
}
{\end{enumerate}}

\newenvironment{3qu}
{
\begin{enumerate}[label=(\roman*)]
}
{\end{enumerate}}

% Normal Environments 
\newenvironment{list0.5}
{
\begin{enumerate}
\setlength\itemsep{0.5em}
}
{\end{enumerate}}

% \titleformat{\section}{\vbox{\rule{\linewidth}{0.8pt}}\bigskip\LARGE\bfseries}{\thesection}{1em}{}

\titleformat{\section}
  {\normalfont\Large\bfseries}{\thesection}{1em}{}[{\titlerule[0.8pt]}] % horizontal line below

% thmtools Environments
\newenvironment{problems}
{
    \subsection{Problems}
    \begin{enumerate}
    \setlength\itemsep{1em}
}
{\end{enumerate}}

\newenvironment{+problems}
{
    \subsubsection{Problems}
    \begin{enumerate}
    \setlength\itemsep{1em}
}
{\end{enumerate}}

\newenvironment{example}[1]
{
    \begin{exampleswap}
        #1
    \end{exampleswap}
    \begin{examplesolution}
}
{
\end{examplesolution}}

\newenvironment{2example}[1]
{
    \begin{exampleswap}[#1]
}
{
\end{exampleswap}}

% Managing Figures
\captionsetup{width=0.8\textwidth}
\renewcommand{\thefigure}{\arabic{chapter}.\arabic{figure}}

% TABLE
% \begin{table}[h!] % delete [h!] if there are bugs

%     %%% TABLE CONFIG %%% 
%     \renewcommand{\arraystretch}{1.5} % changes vertical space for each cell 
%     \setlength{\tabcolsep}{10pt} % changes horizontal space for each cell
%     \setlength{\arrayrulewidth}{0.25mm}

%     \begin{center}
%          title of the table \
%         \vspace{0.5em}
%         \begin{tabular}{|c|c|} % use r, l, c for right, left, center. use m{3cm} for middle width of 3cm, use  b{3cm} for bottom width of 3cm, and use p{3cm} for a top width of 3cm.  
%         \hline
%          &  \ % two columns corresponding to two c's
%         \hline
%          &  \ % second row
%         \hline
%         \end{tabular}
%     \end{center}
%     \caption{}
% \end{table}

\usepackage{pdfpages}

\usepackage{lipsum}
\usepackage{parskip}
\usepackage{titletoc}

\newcommand\circled[1]{
    \begin{tikzpicture}[baseline=(char.base)]%
        \node[circle,draw,inner sep=1pt] (char) {\textsf{#1}};%
\end{tikzpicture}}
% minicircle for in figures!
\newcommand\mc[1]{\footnotesize\circled{#1}}

\usepackage{cmbright}
\usepackage{bm}

% \usepackage{eso-pic}                % put things into background 
% \usepackage{lipsum}                 % for sample text

% \definecolor{reallylightgray}{HTML}{FAFAFA}
% \AddToShipoutPicture{% from package eso-pic: put something to the background
%     \ifthenelse{\isodd{\thepage}}{
%           % ODD page: left bar
%           \AtPageLowerLeft{% start the bar at the left bottom of the page
%             \put(\LenToUnit{\dimexpr\paperwidth-222pt},0){% move it to the top right
%                 \color{reallylightgray}\rule{222pt}{297mm}% }%
%           }%
%       }%
%       {%
%         \AtPageLowerLeft{% put it at the left bottom of the page
%           \color{reallylightgray}\rule{222pt}{297mm}%
%         }%
%    }%
% }



\usepackage{lewis}
\usepackage{bohr}
\usepackage[version=4]{mhchem}
\pagenumbering{gobble}

\declaretheorem[numbered=no, style=thmgreenbox, name=Word Equation]{wordequation}
\declaretheorem[numbered=no, style=thmgreenbox, name=Skeleton Equation]{skeletonequation}
\declaretheorem[numbered=no, style=thmgreenbox, name=Balanced Chemical Equation]{balancedchemicalequation}
\declaretheorem[numbered=no, style=thmgreenbox, name=WHMIS Safety Symbols]{whmis}

\begin{document}
\chapter*{Table Salt}
\vspace{-2em}
Table salt, also know as \emph{sodium chloride}, are crystals that are transparent, colourless, and brittle.\sidecite[]{saltassociation}

A visual representation of salt is seen in Figure \ref{fig:salt}. 


\begin{marginfigure}
    \centering 
    \includegraphics[width=0.5\textwidth]{../figures/table salt.jpg}
    \caption{An image of salt}
    \label{fig:salt}
\end{marginfigure}

\begin{wordequation}
Salt is made from a \textbf{synthesis reaction}, where sodium and chlorine are the reactants and sodium chloride is the product. 
    \[
        \text{sodium}+ \text{chlorine}\to \text{sodium chloride}
    \]
\end{wordequation}

\begin{skeletonequation}
    The skeleton equation for table salt is 
    \[
        \ce{Na_{(s)}}+ \ce{Cl_{2(g)}}\to \ce{NaCl_{(s)}}
    \]
    Where $\ce{Na_{(s)}}$ is sodium and $\ce{Cl_{2(g)}}$ is chlorine (diatomic molecule).
\end{skeletonequation}

\begin{balancedchemicalequation}
    The balanced chemical equation is 
    \[
        \ce{2Na_{(s)}}+ \ce{Cl_{2(g)}}\to \ce{2NaCl_{(s)}}
    \]
\end{balancedchemicalequation}

\begin{marginfigure}
    \centering
    \includegraphics[width=\textwidth]{../figures/road salt.jpg}
    \caption{Road salts are used to melt ice}
    \label{road-salt}
\end{marginfigure}

\textbf{Common Uses}\\
Some common uses for salt are 
\begin{enumerate}
    \item{Cooking; salt is most known for being a food preservative and used as a flavoring agent. It is also used for seasoning foods such as steak, soup, and chicken.}
    \item{Road salt; is used to melt ice to prevent cars and people from slipping (see Figure \ref{road-salt}). The salt compound decomposes into sodium and chlorine ions which disrupt bonds between water molecules.}
    \item{Removing stains; salt has powerful dehydration properties that will eliminate stains such as blood stains.}
\end{enumerate}

\textbf{Environmental Impacts}
\begin{enumerate}
    \item{The chemical reaction that makes table salt doesn't cause emission of any green house gases whatsoever.}
    \item{Table salt can contaminate water by making it too salty, and thus undrinkable. This can harm marine animals and wildlife.}
    \item{Table salt is generally environmentally friendly, unless used in high concentrations. For instance, road salt is damaging to aquatic animals.}
\end{enumerate}

\section*{Reactant: Chlorine}
Chlorine is a yellow-green gas at room temperature that is poisonous at high concentrations.\sidecite[]{chlorine}

\begin{marginfigure}
    \centering
    \vspace{2em}
    \includegraphics[width=0.5\textwidth]{../figures/chlorine_gas.png}
    \caption{Chlorine gas in an erlenmeyer flask}
    \label{fig:chlorine}
\end{marginfigure}

\begin{whmis}
    The WHMIS safety symbols are 

    \centering
    \includegraphics[width=0.7\textwidth]{../figures/whmis.png}
    
\end{whmis}

\textbf{Respective Symbols:} Oxidizing hazard, gas under pressure, acute toxicity, corrosive, can cause damage to environment. \sidecite[]{chlorine-whmis}
\vspace{2em}

\textbf{Chlorine: 3 Safety Tips}\sidecite[]{chlorine-whmis}
\begin{enumerate}
    \item{Do not get in eyes or on skin.}
    \item{Contain gas under pressure.}
    \item{Maintain suitable ventilation.}
\end{enumerate}
\vspace{2em}

\textbf{Chlorine: 2 Pieces of PPE}\sidecite[]{chlorine-whmis}
\begin{enumerate}
    \item{Put on appropriate personal protective equipment, such as a hazmat suit (see Figure \ref{fig:hazmat-suit}).}
    \item{Equip a respiratory when ventilation is poor.}
\end{enumerate}

\begin{marginfigure}
    \centering
    \includegraphics[width=0.5\textwidth]{../figures/hazmat suit.png}
    \caption{Hazmat suit}
    \label{fig:hazmat-suit}
\end{marginfigure}
\vspace{2em}

\textbf{Chlorine: 2 Tips for When it Goes Wrong}\sidecite[]{what-to-do}
\begin{enumerate}
   \item{Escape the area infected by the chlorine as soon as possible and breathe fresh air.}
   \item{Seek higher ground\sidenote{You want to seek higher ground because chlorine is heavier than air, and will thus sink.}, wash your entire body while holding your breathe, and seek help immediately.}
\end{enumerate}

\end{document}
