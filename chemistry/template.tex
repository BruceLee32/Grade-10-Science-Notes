\documentclass[12pt]{report}

%%%%%%%%%%%%%%%%%%%%%%%%%%
%%%%%% UNIT PREAMBLE %%%%% 
%%%%%%%%%%%%%%%%%%%%%%%%%%

% Basics 
\usepackage[utf8]{inputenc}
\usepackage[T1]{fontenc}
\usepackage{textcomp}
\usepackage[a4paper, left=1in, right=1in, top=1in, bottom=1in]{geometry}
\renewcommand\familydefault{\sfdefault}

\usepackage[Sonny]{fncychap}

\usepackage{amsmath,amsfonts,amsthm,amssymb,mathtools}
\usepackage[varbb]{newpxmath}
\usepackage{xfrac}
\usepackage[usenames,dvipsnames]{xcolor} % usenames, dvipsnames adds more colours
\usepackage{hhline}
\usepackage{comment}
\usepackage{tasks}
\usepackage{enumerate} 
\usepackage{enumitem} 
\usepackage{titlesec}
\usepackage[most]{tcolorbox}
\usepackage{lipsum}
\usepackage{tabularx}
\usepackage[labelfont=bf]{caption}
\usepackage{subfig}
\usepackage{pgfplots}
\usepackage{cancel} 
\usepackage{physics} 
\usepackage[bookmarks]{hyperref}
\usepackage{array}
\usepackage{float}
\usepackage{standalone}
\usepackage{graphicx}
\usepackage{forest}

% Tables 
\numberwithin{table}{section}

% Inkscape Figures
\usepackage{import}
\usepackage{xifthen}
\usepackage{pdfpages}
\usepackage{transparent}
\newcommand{\incfig}[2][1]{
    \def\svgwidth{#1\columnwidth}
    \import{../figures/}{#2.pdf_tex}
}

\pdfsuppresswarningpagegroup=1

% Chemistry
\usepackage{lewis} 
\usepackage{bohr}
\usepackage[version=4]{mhchem}

% Page setup
\hypersetup{hidelinks}
\pagenumbering{arabic}
\pagestyle{plain}
\setlength{\parindent}{0pt}

% Show subsubsections
\setcounter{tocdepth}{3}
\setcounter{secnumdepth}{3}

% Required for the Grid
\usetikzlibrary{calc}

% Section Font-Size
\titleformat{\subsubsection}
  {\normalfont\fontsize{12}{12}\bfseries}{\thesubsubsection}{1em}{}

\titleformat{\subsection}
  {\normalfont\fontsize{14}{14}\bfseries}{\thesubsection}{1em}{}

\titleformat{\section}
  {\normalfont\fontsize{16}{16}\bfseries}{\thesection}{1em}{}

% New page for each section 
\newcommand{\sectionbreak}{\clearpage}

% Section Spacing
\titlespacing{\section}{0em}{2.5em}{1em}
\titlespacing{\subsection}{0em}{2.5em}{1em}
\titlespacing{\subsubsection}{0em}{2.5em}{1em}

% TABLE COLUMN SEPARATION (USES ARRAY PACKAGE)
% \renewcommand{\arraystretch}{1.8} % changes vertical space for each cell 
% \setlength{\tabcolsep}{18pt} % changes horizontal space for each cell
% \setlength{\arrayrulewidth}{0.25mm}

% TCOLORBOX 
% \newtcolorbox[auto counter, number within=section]{definition}{colback=white,title=Example~\thetcbcounter,breakable,colframe=white,boxrule=0pt, enhanced, title style={left color=gray!60,right color=white,middle color=white},arc=0mm, titlerule=0pt, fonttitle=\bfseries\sffamily}

% Theorems 
\usepackage{thmtools}
\usepackage[framemethod=TikZ]{mdframed}

\declaretheoremstyle[
    headfont=\bfseries\sffamily\color{ProcessBlue!70!black}, bodyfont=\normalfont,
    headpunct= :,
    mdframed={
        linewidth=2pt,
        rightline=false, topline=false, bottomline=false,
        linecolor=ProcessBlue, backgroundcolor=ProcessBlue!5,
        innerbottommargin=10pt
    } ]{note}

\declaretheoremstyle[
    headfont=\bfseries\sffamily\color{NavyBlue!70!black}, 
    bodyfont=\normalfont,
    % headpunct=,
    mdframed={
        linewidth=2pt,
        rightline=false, topline=false, bottomline=false, linecolor=NavyBlue, innerbottommargin=10pt
    }
]{solution}

\declaretheoremstyle[
    headfont=\bfseries\sffamily\color{Gray!70!black}, bodyfont=\normalfont,
    % headpunct= ,
    postheadspace=\newline,
    mdframed={
        linewidth=2pt,
        rightline=false, topline=false, bottomline=false,
        linecolor=Gray, backgroundcolor=Gray!5,
        innerbottommargin=10pt
    } ]{remark}

\declaretheoremstyle[
    headfont=\bfseries\sffamily\color{Fuchsia!70!black}, bodyfont=\normalfont,
    % headpunct= ,
    mdframed={
        linewidth=2pt,
        rightline=false, topline=false, bottomline=false,
        linecolor=Fuchsia, backgroundcolor=Fuchsia!5,
        innerbottommargin=10pt
    }
]{example}

\declaretheoremstyle[
    headfont=\bfseries\sffamily\color{Fuchsia!70!black}, 
    bodyfont=\normalfont,
    % headpunct= ,
    mdframed={
        linewidth=2pt,
        rightline=false, topline=false, bottomline=false,
        linecolor=Fuchsia,
    }
]{examplesolution}

\declaretheoremstyle[
    headfont=\bfseries\sffamily\color{black!70!black}, 
    bodyfont=\normalfont,
    mdframed={
        linewidth=1pt,
        rightline=false, topline=false, bottomline=false,
        linecolor=black,
    }
]{definition}

\declaretheorem[style=solution, name=Solution, numbered=no]{solution}

\declaretheorem[style=solution, name=Derivation, numbered=no]{derivation}

\declaretheorem[style=definition, name=Definition, numberwithin=chapter]{definition}

\declaretheorem[style=note, name=Note, numbered=no]{noteswap}
\newenvironment{note}[1]{\vspace{0.5em}\begin{noteswap}[#1]}{\end{noteswap}\vspace{0.5em}}

\declaretheorem[style=remark, name=Remark, numbered=no]{remarkswap}
\newenvironment{remark}{\vspace{0.5em}\begin{remarkswap}}{\end{remarkswap}\vspace{0.5em}}

\declaretheorem[style=example, name=Example, numbered=no]{exampleswap}
% \newenvironment{example}{\vspace{0.5em}\begin{exampleswap}}{\end{exampleswap}}

\declaretheorem[style=examplesolution, name=Solution, numbered=no]{examplesolutionswap}
\newenvironment{examplesolution}{\vspace{-2em}\begin{examplesolutionswap}}{\end{examplesolutionswap}}

% Enumerate environments 
\newenvironment{2qu}
{
\begin{enumerate}[label=(\alph*)]
}
{\end{enumerate}}

\newenvironment{3qu}
{
\begin{enumerate}[label=(\roman*)]
}
{\end{enumerate}}

% Normal Environments 
\newenvironment{list0.5}
{
\begin{enumerate}
\setlength\itemsep{0.5em}
}
{\end{enumerate}}

% \titleformat{\section}{\vbox{\rule{\linewidth}{0.8pt}}\bigskip\LARGE\bfseries}{\thesection}{1em}{}

\titleformat{\section}
  {\normalfont\Large\bfseries}{\thesection}{1em}{}[{\titlerule[0.8pt]}] % horizontal line below

% thmtools Environments
\newenvironment{problems}
{
    \subsection{Problems}
    \begin{enumerate}
    \setlength\itemsep{1em}
}
{\end{enumerate}}

\newenvironment{+problems}
{
    \subsubsection{Problems}
    \begin{enumerate}
    \setlength\itemsep{1em}
}
{\end{enumerate}}

\newenvironment{example}[1]
{
    \begin{exampleswap}
        #1
    \end{exampleswap}
    \begin{examplesolution}
}
{
\end{examplesolution}}

\newenvironment{2example}[1]
{
    \begin{exampleswap}[#1]
}
{
\end{exampleswap}}

% Managing Figures
\captionsetup{width=0.8\textwidth}
\renewcommand{\thefigure}{\arabic{chapter}.\arabic{figure}}

% TABLE
% \begin{table}[h!] % delete [h!] if there are bugs

%     %%% TABLE CONFIG %%% 
%     \renewcommand{\arraystretch}{1.5} % changes vertical space for each cell 
%     \setlength{\tabcolsep}{10pt} % changes horizontal space for each cell
%     \setlength{\arrayrulewidth}{0.25mm}

%     \begin{center}
%          title of the table \
%         \vspace{0.5em}
%         \begin{tabular}{|c|c|} % use r, l, c for right, left, center. use m{3cm} for middle width of 3cm, use  b{3cm} for bottom width of 3cm, and use p{3cm} for a top width of 3cm.  
%         \hline
%          &  \ % two columns corresponding to two c's
%         \hline
%          &  \ % second row
%         \hline
%         \end{tabular}
%     \end{center}
%     \caption{}
% \end{table}

% Symbols 
\newcommand{\z}{\mathbb{Z}}

\renewcommand{\l}{\ell}

% Formatting 
\newcommand{\invis}{\vphantom{Invisible Text}}

% Chemistry
% \newcommand{\2ch}[2]{\ce{#1}_{(#2)}}
% \newcommand{\io}[2]{\text{#1}^{#2}} 
% \newcommand{\2io}[3]{\text{#1}^{#2}_{#3}}


\begin{document}
\tableofcontents 

\newpage
\chapter{Chemistry}

\section{Atoms}
Atoms are composed of three subatomic particles:
\begin{enumerate}
\setlength\itemsep{1em}
    \item{Proton (+)}
    \item{Neutron (=)}
    \item{Electron (-)}
\end{enumerate}
Where the proton has a positive charge, the neutron has a neutral charge, and the electron has a negative charge. The proton and neutron are located inside the nucleus of an atom, and the electron is located on the outside of the nucleus (see figure below).  
\begin{figure}[htb!]
\centering
\includegraphics[scale=0.4]{figures/chemistry/atoms/figure1.png}
    \caption{a 3d image of an atom}
\end{figure}

\textbf{Terms:} Here are some terms of the different type of atoms 
\begin{itemize}
    \item{ \textbf{Isotopes:} Atoms that have the same ionic number but different mass numbers; atoms with $\pm$ a certain number of neutrons.}
    \item{ \textbf{Ions:} Atoms that have the same atomic number but a different number of electrons.}
    \item{ \textbf{Cation:} A positively charged ion.}
    \item{ \textbf{Anion:} A negatively charged ion.}
\end{itemize}

\textbf{Note:} Atoms become ions when there is a loss/gain of electrons. Protons do not move.

\newpage
\subsubsection{Examples}
\begin{enumerate}
\setlength\itemsep{0.5em}
    \item{Below is an example of a normal atom vs an ionic atom of sodium
            \begin{table}[h!]
                \renewcommand{\arraystretch}{1.8} % changes vertical space for each cell 
                \setlength{\tabcolsep}{18pt} % changes horizontal space for each cell
                \setlength{\arrayrulewidth}{0.25mm}
                \begin{center}
                    \textbf{Sodium vs Sodium Isotope}\\
                    \vspace{0.5em}
                    \begin{tabular}{|c|c|c|}
                    \hline
                      & $\text{Na}$ & $ \text{Na}^{+}$ \\ % two columns corresponding to two c's
                    \hline
                        Protons & 11 & 11 \\ % second row
                    \hline
                        Electrons & 11 & 10\\
                    \hline 
                        Neutrons & 12 & 12\\
                    \hline
                    \end{tabular}
                \end{center}
            \end{table}
            
            And in this case, we have $ \text{Na}^{+}$ to show that the sodium ion is positively charged, meaning it is missing 1 electron. 
        }
        \item{Below is an example of a normal atom vs an ionic atom of lithium
                \begin{table}[h!] % delete [h!] if there are bugs

                \renewcommand{\arraystretch}{1.8} % changes vertical space for each cell 
                \setlength{\tabcolsep}{18pt} % changes horizontal space for each cell
                \setlength{\arrayrulewidth}{0.25mm}

                    \begin{center}
                        \textbf{Lithium vs Lithium Isotope}\\
                        \vspace{0.5em}
                        \begin{tabular}{|c|c|c|} % use r, l, c for right, left, center. use m{3cm} for middle width of 3cm, use  b{3cm} for bottom width of 3cm, and use p{3cm} for a top width of 3cm.  
                        \hline
                          & $ \text{Li}$ & $ \text{Li}^{+}$ \\ % two columns corresponding to two c's
                        \hline
                            Protons & 3 & 3 \\ % second row
                        \hline
                            Electrons & 3 & 4\\
                        \hline 
                            Neutrons & 3 & 3\\
                        \hline
                        \end{tabular}
                    \end{center}
                \end{table}\\
                And in this case, the isotope of lithium gained an electron so that it is negatively charged. 
            }
\end{enumerate}

\newpage
\section{The Periodic Table}
The periodic table is a table that consists of all the elements in the world of which are sorted by \textit{groups} and \textit{rows}. Additionally, each eleemnt fits into a certain class of element. 

\begin{figure}[htb!]
\centering
\includegraphics[scale=0.2]{figures/chemistry/the periodic table/figure1.png}
    \caption{the periodic table}
\end{figure}

\begin{itemize}
    \item{ \textbf{Group:} A group is a column in the periodic table.}
    \item{ \textbf{Period:} A period is a row in the periodic table.}
\end{itemize}

Each element on the periodic table has the format of an example below 
\begin{figure}[htb!]
\centering
\includegraphics[scale=1]{figures/chemistry/the periodic table/figure4.png}
\end{figure}

And we can list the following: 
\begin{itemize}
    \item{The 7 is the \textbf{atomic number}; number of protons/electrons}
    \item{14.01 is the \textbf{atomic mass}; weight of the atom}
    \item{The element's symbol $ \ce{N}$}
    \item{The name of the eleemnt ```Nitrogen''}
    \item{The ionic charge; the most common charge for when this element becomes an ion. For example, if iron ( $ \ce{Fe}$) has a $2+$ charge, it means that it has lost 2 electrons, since it becomes positive by $2+$}
\end{itemize}

\subsection{Patterns and Trends}
Some patterns and trends in the periodic table are listed below:
\begin{itemize}
    \item{Atomic number steadily increases accross the periodic table.}
    \item{In the periodic table, in period 6 and period 7, you can see how there are lanthanides and actinide which are the last two periods in the peridic table; they just don't fit because it would make the table look very clumped.}
    \item{The number of energy levels (energy shells) steadily increases down a group.}
    \item{Metals tend to LOSE electrons, and therefore they are mostly cations.}
    \item{Non-metals tend to ATTRACT electrons, and therefore they are mostly anions.}
    \item{The number of valence electrons (outermost electrons) are the same within a group and increase steadily across a period.}
\end{itemize}

\subsection{Reactivity}
Valence electrons determine the reactivity of an element and how compounds are formed. Elements tend to lose or gain valence electrons to form bonds and achieve stability.\\

\begin{itemize}
\setlength\itemsep{0.5em}
    \item{Noble gases are stable due to a complete valence shell.}
    \begin{itemize}
        \item{Helium}
        \item{Neon}
        \item{Argon}
    \end{itemize}
    \item{Akali metals are very reactive, since they only need to lose 1 electron to complete their shell.} 
    \item{Halogens are very reactive as well, since they only need to gain 1 electron to complete their shell.}
\end{itemize}

\begin{figure}[H]
\centering
    \includegraphics[scale=0.22]{figures/chemistry/the periodic table/figure2.jpg}
\end{figure}

\subsection{Multivalent Elements}
Some elements on the periodic table are multivalent, meaning that they have a tendency to gain/lose a different number of electrons. For example, iron ($ \ce{Fe}$) is multivalent, having either a tendency to form a $3+$ charge or a $2+$ charge. The reason for this most likely has to do with the outer shell configuration and experimental tests.\\ 

\begin{figure}[H]
\centering
    \includegraphics[scale=1]{figures/chemistry/the periodic table/figure5.png}
    \caption{circled in red shows the $3+$ and $2+$ charge}
\end{figure}

\begin{note}{ }
The ionic charge that comes first is the one that it has a higher tendency to form for. So if you are not sure which ionic charge to pick, pick the first one, which in the figure above would be $3+$. 
\end{note}

Then, how would you write out a multivalent atom? Well, you simply write the element's name and then put in parantheses the roman numerals for the ionic charge. For example, if I wanted iron with a charge of $2+$, I would write Iron(II). If I wanted iron with a charge of $3+$, I would write Iron(III).

\newpage
\section{Bohr-Rutherford Diagrams}
\begin{definition}{Bohr-Rutherford Diagrams}
    Bohr-Rutherford diagrams are visual demonstrations of an atom with its electron shells and the nucleus. 
\end{definition}
Steps to draw a Bohr-Rutherford diagram: 

\begin{list0.5}
    \item{Determine the number of protons, neutrons, and electrons}
    \item{Draw the nucleus, including the number of protons and neutrons}
    \item{Determine the number of electron shells (each complete shell follows the 2-8-8-... pattern)}
    \item{Draw the valence electrons}
\end{list0.5}

\begin{example}{Draw the Bohr-Rutherford diagram for oxygen}
    Oxygen has $8-2=6$ valence electrons. The Bohr-Rutherford diagram is shown below 
    \begin{center}
        \bohr{8}{\small{8p8n}}
    \end{center}
\end{example}

\newpage
\section{Ionic Compounds}
\begin{definition}{Ionic Compounds}
Ionic compounds are formed by the electrostatic interaction of a cation and an anion; electrons are transferred from one element to the other. A good approach to approximate this is by understanding a cation as a metal and an anion as a non-metal, although sometimes we will have to return to the former definition. Ionic compounds are able to be formed because the electrons complete an outer shell.
\end{definition}

\begin{note}{ }
A subtle detail that isn't mentioned in the definition is that the elements of ionic compounds have to be of opposite charge, hence the term ``ionic'' in the name. That is why you need one cation and one anion. Then, if you use the latter definition coupled with the fact that ionic compounds have a complete outer shell, then that explains how they derive the \textit{ionic charge.} For example, we see that sulfur has an ionic charge of $-2$, meaning that it has a tendency to gain 2 electrons. This means that sulfur has 6 valence electrons, which can be verified by its atomic number, which explains why it has an ionic charge of $-2$. Then, in order for an ionic compound to have a full outer shell, one of the ions have to be positive and the other has to be negative; hence, positive attracts negative. 
\end{note}

\begin{figure}[htb!]
\centering
    \includegraphics[scale=0.5]{figures/chemistry/ionic compounds/figure1.png}
    \caption{example of an ionic compound}
\end{figure}

\subsection{Physical Properties of Ionic Compounds}
Some physical properties are: 
\begin{itemize}
    \item{High melting and boiling points.}
    \item{Low volatility/non-volatile.}
    \item{Generally soluable in polar solvents (such as water).}
    \item{Does not conduct electricity in the solid state.}
    \item{Conducts electricity when molten or when dissolved in water.}
    \item{Generally brittle; hard but can shatter/break easily.}
\end{itemize}

\begin{table}[h!] % delete [h!] if there are bugs
    \begin{center}
        
        \renewcommand{\arraystretch}{1.5} % changes vertical space for each cell 
        \setlength{\tabcolsep}{10pt} % changes horizontal space for each cell
        \setlength{\arrayrulewidth}{0.25mm}

        Examples \\
        \vspace{0.5em}
        \begin{tabular}{|c|c|c|} % use r, l, c for right, left, center. use m{3cm} for middle width of 3cm, use  b{3cm} for bottom width of 3cm, and use p{3cm} for a top width of 3cm.  
        \hline
        \textbf{Ionic Compound} & \textbf{Charge on Metal Ion} & \textbf{Melting Point}\\ % two columns corresponding to two c's
        \hline
        $ \ce{Na2O}$ & $1-$ & $1132 ^{\circ}C$ \\ % second row
        \hline
        $ \ce{MgO}$ & $2+$ & $2800 ^{\circ}C$\\
        \hline
        \end{tabular}
    \end{center}
\end{table}

\subsection{Naming Ionic Compounds}
This is where the naming for compounds comes in. We use the following steps when naming an ionic compound: 
\begin{enumerate}
\setlength\itemsep{0.5em}
    \item{Make sure that the elements are not multivalent. If they are, un criss-cross the compound (see \ref{section:crisscross}).}
    \item{Name the cation first. The name of the cation is unchanged. For example, potassium stays as potassium.}
    \item{Name the anion second. When the anion becomes part of an ionic compound, you let the anion end in ``ide''. For example, chlorine becomes chloride.}
    \item{Combine the names from step (1) and (2). The name of $ \ce{KCl}$ is therefore Potassium Chloride.}
    \item{If the compound contains a \textit{polyatomic ion}, do not apply step (2) to it.}
\end{enumerate}

\subsubsection{Problems}
\begin{enumerate}
\setlength\itemsep{0.5em}
    \item{Write the chemical name for the following:}
        \begin{enumerate}[label=(\alph*)]
            \item{$ \ce{Fe2O3}$}
            \item{ $ \ce{Cu2S}$}
            \item{ $ \ce{FeN}$}
        \end{enumerate}
    \textit{\textbf{Solutions:}}
    \begin{enumerate}[label=(\alph*)]
        \item{Iron is multivalent, so if we un criss-cross this
                \[
                    \ce{Fe2O3}= \text{Fe}^{3+} \text{O}^{2-}
                \]
            Therefore our answer is iron(III) oxide.}
        \item{Copper is multivalent, so if we un criss-cross this
                \[
                    \ce{Cu2S}= \text{Cu}^{1+} \text{S}^{2-}
                \]
            Therefore, our answer is copper(I) sulfide.}
        \item{Iron is multivalent, so we should just be able to un criss-cross this. However, if we un criss-cross this, we get $ \text{Fe}^{+} \text{N}^{-}$, which doesn't make any sense, since iron can never have an ionic charge of $1+$. Therefore, we have to do some thinking. We recall that we should simplify the subscripts to the smallest ratio whenever we can. Therefore, we can actually write 
                \begin{align*}
                    \ce{FeN}&= \ce{Fe3N3}\\
                            &= \text{Fe}^{3+} \text{N}^{3+}
                \end{align*}
            Where we use the subscript of 3 since nitrogen has an ionic charge of $3-$. Therefore, our answer is iron(III) nitride. 
            }
    \end{enumerate}
    \item{Write the chemical name for the following:}
        \begin{enumerate}[label=(\alph*)]
            \item{ $\ce{KNO3}$}
            \item{ $\ce{(NH4)3PO4}$}
            \item{ $\ce{PbCO3}$}
        \end{enumerate}
        \textit{\textbf{Solutions:}}  
        \begin{enumerate}[label=(\alph*)]
            \item{None of these ions are multivalent, so we just get potassium nitrate.}
            \item{This is actually a compound of two polyatomic ions, so our answer is just ammonium phosphate.}
            \item{Lead is multivalent with a higher tendency to form as a $2+$ ion, so we un criss-cross 
                    \begin{align*}
                        \ce{PbCO3}&= \ce{Pb2(CO3)2}\\
                                  &= \text{Pb}^{2+} \text{CO3}^{2-}_{3}
                    \end{align*}
                And so our final answer is lead(II) carbonate.}
        \end{enumerate}
\end{enumerate}

\subsection{Lewis Dot Diagrams}
\begin{definition}{Lewis Dot Diagrams}
    Lewis Dot Diagrams are diagrams that show the valence electrons of a Bohr-Rutherford Diagram. Creating Lewis Dot Diagrams: 
    \begin{enumerate}
    \setlength\itemsep{0.5em}
        \item{Write the element symbol.}
        \item{Find the number of valence electrons by looking at the group number. For groups 13-18, subtract 10 to obtain the valence electrons.}
        \item{Place dots representing the valence electrons on the four sides of the element symbol solo before pairing up.}
    \end{enumerate}
\end{definition}

\begin{example}{Draw a Lewis Dot Diagram of a chlorine atom.}
    Chlorine has 7 valence electrons, so we represent those 7 electrons with 7 dots:
    \begin{center}
        \lewis{Cl}{.}{.}{.}{.}{.}{.}{.}{}
    \end{center}
\end{example}

\begin{+problems}
    \item{Draw the Lewis Dot Diagrams of:}
        \begin{enumerate}[label=(\alph*)]
           \item{Silicon}
           \item{Calcium}
           \item{Magnesium ion $ \text{Mg}^{2+}$}
           \item{Oxide ion $ \text{O}^{2-}$}
        \end{enumerate}
    \begin{solution}
       \begin{enumerate}[label=(\alph*)]
           \item{Silicon has $14-10=4$ valence electrons. The Lewis Dot Diagram is 
                   \begin{center}
                      \lewis{Si}{.}{.}{.}{.}{}{}{}{} 
                   \end{center}
                   
               }
            \item{Calcium has $2$ valence electrons. The Lewis Dot Diagram is 
                    \begin{center}
                        \lewis{Ca}{.}{.}{}{}{}{}{}{}
                    \end{center}
                }
            \item{The magnesium ion has no valence electrons $(12-10)-2=0\to8 \text{valence electrons}$. The Lewis Dot Diagram for ions is 
                    \[
                        \left[ \lewis{Mg}{.}{.}{.}{.}{.}{.}{.}{.} \right]^{2+}
                    \]
                }
            \item{The oxygen $2-$ (oxide) ion has $(8-2)+2=8$ valence electrons. The Lewis Dot Diagram is 
                    \[
                        \left[ \lewis{O}{.}{.}{.}{.}{.}{.}{.}{.} \right]^{2-}
                    \]
                }
       \end{enumerate} 
    \end{solution}
\end{+problems}

\subsection{The Criss-Cross Method}\label{section:crisscross}
This method is used to determine the chemical formula for ionic compounds. The criss-cross method is simple: you write the elements with their ionic charges next to each other, and then you criss-cross the powers down into positive subscripts.\\ 

\textbf{Note:} Once you have written the subscripts for the criss-cross method, you have to simplify the ratio between the subscripts to the lowest. For example, if I had $ \ce{ \text{Ti}^{4+} \text{O}^{2-}= \ce{Ti2O4}}$, I would simplify this to be $ \ce{TiO2}$.\\

\textbf{Examples}
\begin{enumerate}
\setlength\itemsep{0.5em}
\item{The chemical formula for calcium chloride is as follows: calcium has the symbol $\ce{Ca}$ with an ionic charge of $2+$, and oxygen has the symbol $Cl$ with an ionic charge of $1-$. Therefore, the chemical formula using the criss-cross method will be 
           \[
               \text{Ca}^{2+} \text{Cl}^{1-}= \ce{Ca1Cl2}= \ce{CaCl2}
           \]
       }

    \item{The chemical formula for water (hydrogen dioxide) is as follows: hydrogen has the symbol $ \ce{H}$ with an ionic charge of $1+$, and oxygen has the symbol $ \ce{O}$ with an ionic charge of $2-$. Therefore, the chemical formula using the criss-cross method will be  
            \[
                \text{H}^{1-} \text{O}^{2-}= \ce{H2O1}= \ce{H2O}
            \]
        }

        \item{The chemical formula for boron nitride is as follows: boron has the symbol $ \ce{B}$ with an ionic charge of $3+$, and nitrogen has the symbol $ \ce{N}$ with an ionic charge of $3-$. Therefore, the chemical formula using the criss-cross method will be 
                \[
                    \text{B}^{3+} \text{N}^{3-}= \ce{B3N3}= \ce{BN}
                \]
            Where we simplify the ratio $3:3$ to be $1:1$. 
            }
\end{enumerate}

\newpage 
\section{Covalent Compounds}
\begin{definition}{Covalent Compounds}
A covalent compound is formed by a covalent bond between two elements, typically non-metals. Covalent bonding is basically the same thing as ionic bonding except that we have to forget about all the intuition behind the chemical formulas for ionic compounds. Covalent bonding is simply the regular type of bonding for atoms except that they share the electrons instead of giving them away.
\end{definition}

\begin{note}{ }
    You do not simplify the subscripts for covalent compounds. 
\end{note}

\subsection{Physical Properties of Covalent Compounds}
\begin{itemize}
    \item{Lower melting points and boiling points than ionic compounds}
    \item{Some of them are volatile (diatomic fluorine, chlorine, bromine)}
    \item{Not very soluable in water}
    \item{Doesn't conduct electricity very well}
\end{itemize}

\subsection{Naming Covalent Compounds}
Naming covalent compounds is basically the same as naming ionic compounds, only that you have prefixes for the number of elements in a compound.\\ 

Steps for naming covalent compounds:
\begin{list0.5}
    \item{Write the name of the elements}
    \item{Change the ending of the second element to "-ide"}
    \item{Add prefixes to represent the number of atoms}
    \begin{enumerate}[label=(\alph*)]
        \item{If mono- is the first prefix, it is understood and not written}
        \item{If the element starts with a vowel, exclude part (a) on the prefix.}
    \end{enumerate}
\end{list0.5}

\begin{table}[h!] % delete [h!] if there are bugs

    %%% TABLE CONFIG %%% 
    \renewcommand{\arraystretch}{1.5} % changes vertical space for each cell 
    \setlength{\tabcolsep}{10pt} % changes horizontal space for each cell
    \setlength{\arrayrulewidth}{0.25mm}

    \begin{center}
        Prefixes \\
        \vspace{0.5em}
        \begin{tabular}{|c|c|} % use r, l, c for right, left, center. use m{3cm} for middle width of 3cm, use  b{3cm} for bottom width of 3cm, and use p{3cm} for a top width of 3cm.  
        \hline
        \textbf{Number} & \textbf{Term} \\ % two columns corresponding to two c's
        \hline
        1 & Mono \\ % second row
        \hline
        2 & Di\\ 
        \hline 
        3 & Tri\\ 
        \hline 
        4 & Tetra\\ 
        \hline
        5 & Penta\\
        \hline 
        6 & Hexa\\ 
        \hline 
        7 & Hepta\\ 
        \hline 
        8 & Octa\\ 
        \hline 
        9 & Nona\\ 
        \hline 
        10 & Deca\\
        \hline
        \end{tabular}
    \end{center}
\end{table}

\begin{example}{Determine the chemical name for $\ce{PF5}$}
   This compound has phosphorus and fluorine. Since there is 1 phosphorus and 6 fluorines, the answer will be Mono Phosphorus Penta Fluorine. 
\end{example}

\subsection{Polyatomic Ions}
\begin{definition}{Polyatomic Ions}
A polyatomic ion is an ion that is made of more than one atom but acts as a single unit. When criss-crossing polyatomic ions, do \textbf{NOT} simplify the subscripts. 
\end{definition}

For example, nitrate acts as a single unit and has the chemical formula of $ \text{NO}^{-}_{3}$. Another example could be hydroxide, which has a chemical formula of $ \text{OH}^{-}.$\\

\begin{remark}{ } 
One thing you may notice about these polyatomic ions is that they are all made by covalent bonds. We can see this because they are all made of anions, or vastly non-metals. This means that the elements that make up a polyatomic ion share their electrons and hence they act as one unit. This also gives insight as to why you can use the criss-cross method for polyatomic ions as well, since you are just treating that polyatomic ion as an entire compound. 
\end{remark}

\begin{example}{Determine the polyatomic ion of phosphorus nitrate}
    Phosphorus is $\text{P}^{3-}$ and nitrate is $\text{NO}^{-}_{2}$. Using the criss-cross method 
    \[
        \text{P}^{3-} \text{NO}^{-}_{2}\to \ce{P} \ce{(NO2)3} 
    \]
\end{example}

\begin{problems}
    \item{Write the chemical formula for calcium sulfate.}
        \begin{solution}
            Calcium ion is $\text{Ca}^{2+}$ and sulfate is $\text{SO}^{2-}_{4}$. The chemical formula would be 
            \[
                \text{Ca}^{2+} \text{SO}^{2-}_{4}\to \ce{Ca2} \ce{(SO4)2}
            \]
        \end{solution}
    \item{Write the chemical formula for calcium chlorate.}
        \begin{solution}
            Calcium ion is $\text{Ca}^{2+}$ and chlorate is $\text{ClO}^{-}_{3}$. The chemical formula would be 
            \[
                \text{Ca}^{2+} \text{ClO}^{-}_{3}\to \ce{Ca(ClO3)2} 
            \]
        \end{solution}
    \item{Write the chemical formula for iron(III) phosphate.}
        \begin{solution}
            Iron(III) ion is $\text{Fe}^{3+}$ and phosphate is $\text{PO}^{3-}_{4}$. The chemical formula would be 
            \[
                \text{Fe}^{3+} \text{PO}^{3-}_{4}\to \ce{Fe3(PO4)3}
            \]
            Where we don't simplify it, since it includes a polyatomic ion. 
        \end{solution}
\end{problems}

\newpage 
\section{Types of Chemical Reactions}
There are 5 major types of chemical reactions: 
\begin{enumerate}
    \setlength\itemsep{0.75em}
    \item{Synthesis Reaction}
    \item{Decomposition Reaction} 
    \item{Combustion Reaction} 
    \item{Single Replacement/Displacement Reaction} 
    \item{Double Replacement/Displacement Reaction} 
\end{enumerate}
Which we will discuss in this particular order. 

\subsection{Synthesis Reaction}
\textbf{Definition:}  A synthesis reaction is a reaction  that describes the process of a compound that is made from simpler materials. In some sense, ``synthesis'' is just a fancy word for ``making''.\\ 

\textbf{Note:} Synthesis reactions are also sometimes called \textit{combination reactions}.\\

The general skeleton equation is 
\begin{center}
    \framebox{$\displaystyle \ce{A}+\ce{B}\to \ce{AB}$}
\end{center}

Where $A$ and $B$ are arbitrary ``materials''.\\

\subsubsection{Examples}
\begin{enumerate}
\setlength\itemsep{0.5em}
    \item{Carbon comes together with oxygen gas, to form carbon dioxide
            \[
                \ce{C(s)}+ \ce{O2(g)}\to \ce{CO2(g)}
            \]
        And we see that in this scenario, $ CO2$ is more complex than the two simple things that we started with.  
        }
    \item{Sodium comes together with chlorine gas to make sodium chloride 
            \[
                \ce{2Na(s)}+ \ce{Cl2(g)}\to \ce{2NaCl(s)}
            \]
        And what we end up with is more complex than what we started with.\\
        }
    
\end{enumerate}

\subsubsection{Problems}
\begin{enumerate}
\setlength\itemsep{1em}
    \item{Lithium and chlorine combine to produce lithium chloride. Identify the type of chemical reaction, and then write the balanced chemical equation.}\\

        \textit{ \textbf{Solution:} }Lithium has the symbol $ \ce{Li}$ and chlorine has the symbol $ \ce{Cl}$. Lithium chloride has the chemical formula $ \text{Li}^{+} \text{Cl}^{-}= \ce{LiCl}$. Then, we recognize that the reaction is simply a synthesis reaction, because there are only two elements involved and the chemical reaction forms a compound. Hence, we get 
        \begin{align*}
            \ce{Li}+ \ce{Cl} \to \ce{LiCl}
        \end{align*}
    Which is luckily already balanced.

    \item{Calcium oxide is produced from calcium and oxygen. Identify the type of chemical reaction, and then write the balanced chemical reaction.}\\

        \textit{ \textbf{Solution:}} Calcium has the symbol $ \ce{Ca}$ and oxygen has the symbol $ \ce{O}$. Calcium oxide has the chemical formula $ \text{Ca}^{2+} \text{O}^{2-}= \ce{Ca2O2}= \ce{CaO}$. Then, we recognize that this is a synthesis reaction, since there are two elements that form a compound. Therefore, our chemical equation is 
        \[
            \ce{Ca}+ \ce{O}\to \ce{CaO}
        \]
    Which is luckily already balanced.

    \item{Carbon dioxide gas can be formed from two non-metals: solid carbon and oxygen gas. Identify the type of chemical reaction, and then write the balanced chemical reaction, including states.}\\

        \textit{ \textbf{Solution:}} Solid carbon has the symbol $C_{\text{(s)}}$ and oxygen gas has the symbol $O_{\text{(g)}}$. Then, carbon dioxide has the chemical formula $ \text{Ca}^{4+} \text{O}^{2-}= \ce{Ca2O4}= \ce{CaO2}$.\\

    \textbf{Note:} Carbon could really have an ionic charge of $4+$ or $4-$; it really depends on the situation. This is of course because carbon has 4 valence electrons. However, in our case, since ionic compounds can only be formed with a cation and anion, we choose carbon to have a charge of $4+$.\\

        Then, our chemical equation will become 
        \begin{align*}
            \ce{Ca_{(s)}}+ \ce{O_{(g)}} &\to \ce{CaO2_{(g)}}\\
            \ce{Ca_{(s)}}+ \ce{2O_{(g)}} &\to \ce{CaO2_{(g)}}
        \end{align*}
        Which we balance. 

\end{enumerate}

\subsection{Decomposition Reaction}
\textbf{Definition:} A process in which a compound is broken down into simpler compounds, or all the way down to the elements that make it up. In other words, the opposite of a synthesis reaction.\\

The general chemical equation is 

\begin{center}
    \framebox{ $\displaystyle \ce{AB}\to \ce{A}+ \ce{B}$}
\end{center}

Where $AB$ is a compound and $A$ and $B$ are simpler compounds or elements.\\

\subsubsection{Examples}
\begin{enumerate}
\setlength\itemsep{0.5em}
    \item{ Water is broken down into hydrogen and oxygen
            \[
                \ce{2H2O( \ell)}\to \ce{2H2(g)}+ \ce{O2(g)}
            \]
        }
    \item{However, not all compounds have to be broken down into their basic elements; they can be broken down into simpler compounds instead. For example, calcium carbonate is broken down into carbon dioxide and calcium oxide
            \[
                \ce{CaCO3(s)}\to \ce{CaO(s)}+ \ce{CO2(g)}
            \]
        }
    
\end{enumerate}

\subsubsection{Problems}
Solid calcium chloride is used to reduce dust on roads and in mines. When calcium chloride is heated to a sufficiently high temperature, it undergoes a chemical reaction that produces solid calcium and chlorine gas. Identify the type of chemical reaction, and then write the balanced chemical equation, including states.
\begin{enumerate}
\setlength\itemsep{1em}
\item{Write the word equation for the reaction.}\\

    \textit{ \textbf{Solution:}} We know that this is a decomposition reaction since we are beginning with a more complex compound calcium chloride and then breaking it down into calcium and chlorine, separately. Therefore, the answer is simply
    \[
        \text{calcium chloride}\to \text{calcium}+ \text{chlorine}
    \]

\item{Identify the type of chemical reaction from the nature of the reactants and products.}\\

    \textit{\textbf{Solution:}} Once again, this is a decomposition reaction.  

\item{Use the word equation to write the skeleton equation.}\\

    \textit{\textbf{Solution:}} Calcium has the symbol $ \ce{Ca_{(s)}}$ and chlorine has the symbol $ \ce{Cl_{(g)}}$. Then, calcium chloride has the chemical formula $ \text{Ca}^{2+} \text{Cl}^{1-}= \ce{CaCl2}$. Therefore, the skeleton equation is 
    \[
        \ce{CaCl2_{(s)}}\to \ce{Ca_{(s)}}+ \ce{Cl_{(g)}}
    \]

\item{Write the balanced chemical equation for the reaction.}\\

    \textit{\textbf{Solution:}} The balanced skeleton equation is simply 
    \begin{align*}
        \ce{CaCl2_{(s)}}&\to \ce{Ca_{(s)}}+ 2\ce{Cl_{(g)}}\\
        \ce{CaCl2_{(s)}}&\to \ce{Ca_{(s)}}+ \ce{Cl2_{(g)}}
    \end{align*}
    Where either forms work: $\ce{2Cl_{(g)}}= \ce{Cl2_{(g)}}$.

    \item{During a chemical reaction, magnesium and sulfur are produced from magnesium sulfide. Identify the type of chemical reaction, and then write the balanced chemical equation.}\\

    \textit{\textbf{Solution:}} The type of chemical reaction is simply decomposition. The symbol for magnesium is $ \ce{Mg}$ and the symbol for sulfur is $ \ce{S}$. The chemical formula for magnesium sulfide is $ \text{Mg}^{2+} \text{S}^{2-}= \ce{Mg2S2}= \ce{MgS}$. Therefore, the balanced chemical equation will be 
    \[
        \ce{MgS}\to \ce{Mg}+ \ce{S}
    \]

    \item{When heated, sodium iodide may be broken down into sodium and iodine. Identify the type of chemical reaction, and then write the balanced chemical equation.}\\

    \textit{\textbf{Solution:}} The type of chemical reaction is not combustion but instead still decomposition. The symbol for sodium is $ \ce{Na}$ and the symbol for iodine is $ \ce{I}$. The chemical formula for sodium iodide is $ \text{Na}^{1+} \text{I}^{1-}= \ce{NaI}$. Therefore, the balanced chemical equation will be 
    \[
        \ce{NaI}\to \ce{Na}+ \ce{I}
    \]
    
    \item{Chlorine gas is used to disinfect water in drinking supplies and pools, and in making plastics, pharmaceuticals, and fertilizers. In the lab, chlorine gas and sodium metal can be produced by passing an electric current through hot, molten, sodium chloride, which is a liquid. Identify the type of chemical reaction, and then write the balanced chemical equation, including states.}\\

    \textit{\textbf{Solution:}} We are given that chlorine $ \ce{Cl}$ is a gas, sodium $ \ce{Na}$ is a metal and sodium chloride $ \text{Cl}^{1-} \text{Na}^{1+}= \ce{NaCl}$ (don't forget to write the cation first). Therefore, the balanced chemical equation will be 
    \[
        \ce{NaCl_{( \ell )}}\to\ce{Cl_{(g)}}+ \ce{Na_{(s)}}
    \]

\end{enumerate}

\subsection{Combustion Reaction}
\textbf{Definition:} A process in which a compound containing carbon and hydrogen (and sometimes oxygen) combines with oxygen gas to produce carbon dioxide and water. Combustion is also a fancy word for burning, hence we have carbon dioxide and water. \\

The general chemical equation for a \textit{complete combustion} is 

\begin{center}
        \framebox{ $ \ce{C_xH_y(O_z)}+ \ce{O2}\to \ce{CO2}+ \ce{H2O}$}
\end{center}

Where we SOMETIMES have oxygen. Additionally, $ \ce{C_xH_y}$ is a hydrocarbon molecule. We also write $ \ce{C_xH_y}$ in the general chemical equation because there are many different compounds made up of only carbon and hydrogen atoms. The products of hydrocarbon combustion reaction are always carbon dioxide and water. \\

The general chemical equation for an \textit{incomplete combustion} is 
\begin{center}
        \framebox{ $ \ce{C_xH_y(O_z)}+ \ce{O2}\to \ce{CO2}+ \ce{H2O}+( \ce{C})+ (\ce{ CO})$}
\end{center}

Where $\ce{C}$ and $\ce{CO}$ aren't always there.

\subsubsection{Examples}
\begin{enumerate}
\setlength\itemsep{0.5em}
    \item{Methane $ \ce{CH4(g)}$ combines with oxygen to form carbon dioxide and water
            \[
                \ce{CH4(g)}+ 2\ce{O2(g)}\to \ce{CO2(g)}+ 2\ce{H2O(g)} 
            \]
        }
    \item{Propane $ \ce{C3H8(g)}$ is added with oxygen to form carbon dioxide and water 
            \[
                \ce{C3H8(g)}+ \ce{3O2(g)}\to \ce{3CO2(g)}+ \ce{4H2O( \ell )}
            \]
        }
    \item{Ethanol $ \ce{C2H5OH( \ell )}$ is mixed with oxygen to form carbon dioxide and propane 
            \[
                \ce{C2H5OH( \ell )}+ \ce{3O2(g)}\to \ce{2CO2(g)}+ \ce{3H2O(g)}
            \]
        }
\end{enumerate}

\subsubsection{Problems}
\begin{enumerate}
\setlength\itemsep{0.5em}
    \item{Determine the chemical reaction and balance the chemical equation for the following skeleton equation
            \[
                \ce{CH4}+ \ce{O2}\to \ce{CO2}+ \ce{H2O}
            \]
        }
    \textit{\textbf{Solution:}} We can readily see that this is a combustion reaction. Then, the balanced chemical equation is simply 
    \[
        \ce{CH4}+ \ce{2O2}\to \ce{CO2}+ \ce{2H2O}   
    \]

    \item{Given that the fuel of a combustion reaction is $ \ce{C2H5OH}$, determine the balanced chemical equation.}\\

        \textit{\textbf{Solution:}} Another form for a combustion reaction is 
        \[
            \text{(fuel)}+ \ce{O2}\to \ce{CO2}+ \ce{H2O}
        \]
        And since we know what the fuel is, the skeleton equation is 
        \[
            \ce{C2H5OH}+ \ce{O2}\to \ce{CO2}+ \ce{H2O}
        \]
        Then, we can balance this equation 
        \begin{align*}
            \ce{C2H5OH}+ \ce{O2}&\to \ce{2CO2}+ \ce{H2O}\\
            \ce{C2H5OH}+ \ce{O2}&\to \ce{2CO2}+ \ce{3H2O}\\
            \ce{C2H5OH}+ \ce{3O2}&\to \ce{2CO2}+ \ce{3H2O}
        \end{align*}
        And this chemical equation is balanced. 
\end{enumerate}

\subsection{Single Replacement/Displacement Reaction}
\textbf{Definition:} A reaction in which one element is substituted for another element in a compound.\\

The general chemical equation is 

\begin{center}
    \framebox{ $ \ce{A}+ \ce{BC}\to \ce{B}+ \ce{AC}$}
\end{center}
For metals switching with metals. For non-metals switching with non-metals: 
\begin{center}
    \framebox{$ \ce{A}+ \ce{BC}\to \ce{C}+ \ce{BA}$}
\end{center}

\textbf{Note:} One thing that should be noted is that when A and C swap places, C is what we call a \textbf{diatomic molecule}. ``Di-atomic'' means consisting of two atoms, meaning that it will have a subscript of 2. Diatomic molecules are usually found in nature. The periodic table below shows elements that are diatomic\\ 

\begin{figure}[htb!]
\centering
    \includegraphics[scale=0.5]{figures/chemistry/types of chemical reactions/diatomic-elements.jpg}
    \caption{The elements in yellow are diatomic}
\end{figure}

\subsubsection{Examples}
\begin{enumerate}
\setlength\itemsep{0.5em}
    \item{Iron combines with copper chloride $ \ce{CuCl2(aq)}$ to make copper and iron chloride $ \ce{FeCl2}$
            \[
                \ce{Fe(s)}+ \ce{CuCl2(aq)}\to \ce{Cu(s)}+ \ce{FeCl2(aq)}
            \]
        }
    \item{Copper combines with silver nitrate $ \ce{AgNO3(aq)}$ to make silver and copper nitrate $ \ce{Cu(NO3)2}(aq)$
            \[
                \ce{Cu(s)}+ \ce{2AgNO3(aq)}\to \ce{2Ag(s)}+ \ce{Cu(NO3)2(aq)}
            \]
        }
    
\end{enumerate}

\subsubsection{Problems}
\begin{enumerate}
\setlength\itemsep{0.5em}
    \item{When magnesium metal and zinc nitrate react, they form zinc and magnesium nitrate. Identify the type of chemical reaction, and then write a balanced chemical equation for the reaction.}\\

    \textit{\textbf{Solution:}} The type of chemical reaction is obviously single replacement/displacement. Before we write the skeleton equation, we have to first determine the chemical formulas for zinc nitrate and magnesium nitrate.\\

    For zinc nitrate, we get $ \text{Zn}^{2+} \text{NO}^{-}_{3}= \ce{Zn(NO3)2}$.\\

    For magnesium nitrate, we have $ \text{Mg}^{+2} \text{NO}^{-}_{3}= \ce{Mg(NO3)2}$.\\

    Then, the skeleton equation becomes 
    \[
        \ce{Mg}+ \ce{Zn(NO3)2}\to \ce{Zn}+ \ce{Mg(NO3)2}
    \]
    Which is luckily already balanced. We also see that magnesium and zinc, which both have a positive ionic charge, swap places.  

    \item{Iron metal is placed into a solution of silver nitrate and allowed to sit. This produces aqeuous iron(II) nitrate and solid silver metal. Identify the type of chemical reaction, and then write a balanced chemical equation for the reaction, including states.}\\
    
    \textit{\textbf{Solution:}} When we do the single displacement, we have to remember to calculate the chemical formulas for the ionic compounds. Therefore, the skeleton equation is 
    \begin{align*}
        \ce{Fe_{(s)}}+ \text{Ag}^{1+}_{} \text{NO}^{-}_{\text{3(aq)}}&\to \text{Fe}^{2+} \text{NO}^{-}_{3( \text{aq} )}+ \ce{Ag_{(s)}}\\
        \ce{Fe_{(s)}}+ \ce{AgNO3_{( aq )}}&\to \ce{Fe(NO3)2_{(aq)}}+ \ce{Ag_{(s)}}
    \end{align*}
    And the balanced chemical equation will be 
    \[
        \ce{Fe_{(s)}}+ \ce{2AgNO3_{(aq)}}\to \ce{Fe(NO3)_{\text{2(aq)}}}+ \ce{2Ag_{(s)}}
    \]

    \item{Predict the products of the following single displacement reaction
            \[
                \ce{Fe}+ \ce{Pb(NO3)2}\to\quad ?
            \]
        }
        \textit{\textbf{Solution:}} We see that iron and lead are both metals, so those are the ones we will be displacing. This gives us
        \begin{align*}
            \ce{Fe}+ \ce{Pb(NO3)2}&\to \text{Fe}^{3+} \text{NO}^{1-}_{3}+\ce{Pb}\\ 
            \ce{Fe}+ \ce{Pb(NO3)2}&\to \ce{Fe(NO3)3}+\ce{Pb}
        \end{align*}
        And the balanced chemical equation will simply be 
        \[
           \ce{3Fe}+ \ce{2Pb(NO3)2}\to \ce{2Fe(NO3)3}+ \ce{3Pb} 
        \]

    \item{Predict the products
            \[
                \ce{Cl2}+ \ce{NaI}\to\quad?
            \]
        }

        \textit{\textbf{Solution:}} This is a single displacement reaction between two non-metals. This means that the non-metal that is by itself after the swap will be diatomic. Hence, we get  
        \begin{align*}
            \ce{Cl2}+ \ce{NaI}&\to \text{Na}^{+} \text{Cl}^{-}+ \ce{I2}\\
            \ce{Cl2}+ \ce{NaI}&\to \ce{NaCl}+ \ce{I2}
        \end{align*}
        And upon balancing 
        \[
            \ce{Cl2}+ \ce{2NaI}\to \ce{2NaCl}+ \ce{I2} 
        \]

    \item{Predict the product and balance
            \[
                \ce{Li_{(s)}}+ \ce{H2O_{( \ell )}}\to\quad?
            \]
        }
        \textit{\textbf{Solution:}} In order to solve this one, we will have to sort of redefine the definition and be more specific. Instead of swapping metals with metals and non-metals with non-metals, we will instead be swapping positively charged molecules with postively charged molecules and swapping negatively charged molecules with negatively charged molecules.\\

        If we use the latter approach, 
    
    \item{Predict the products, balance, and classify
        \[
            \ce{Copper(II) Chloride}+ \ce{Aluminum}\to\quad?
        \]}

        \textit{\textbf{Solution:}} This is a single displacement reaction. The skeleton equation and balanced chemical equation will be 
        \begin{align*}
            \text{Cu}^{2+} \text{Cl}^{-}+ \ce{Al}&\to \text{Al}^{3+} \text{Cl}^{-}+ \ce{Cu}\\
            \ce{CuCl2}+ \ce{Al}&\to \ce{AlCl3}+ \ce{Cu}\\
            \ce{3CuCl2}+ \ce{2Al}&\to \ce{2AlCl3}+ \ce{3Cu}
        \end{align*}
\end{enumerate}


\subsection{Double Replacement/Displacement Reaction}
\textbf{Definition:} A reaction in which the positive and negative ions in two compounds switch places.\\

The general chemical equations are 
\begin{center}
    \framebox{$ \ce{AB}+ \ce{CD}\to \ce{CB}+ \ce{AD}$}
\end{center}
For the cations switching
\begin{center}
    \framebox{$ \ce{AB}+ \ce{CD}\to \ce{AD}+ \ce{CB}$}
\end{center}
For the anions switching. For most cases, you will view cations as metals and anions as non-metals.\\ 

\textbf{Note:} In order for a double displacement reaction to occur, there must be the formation of either products: \textbf{water}, \textbf{a precipitate}, or \textbf{a gas}.\\

A precipitate is a solid formed from the reaction of two solutions (aqeuous). We consider a precipitate to be an insoluble compound, meaning it can't be melted.

\subsubsection{Examples}
\begin{enumerate}
\setlength\itemsep{0.5em}
    \item{We begin with $ \ce{BaCl2(aq)}$ and $ \ce{Na2SO4(aq)}$ together and the $ \ce{Cl}$ and $ \ce{SO4}$ just simply switch places
            \begin{figure}[htb!]
            \centering
                \includegraphics[scale=0.3]{figures/chemistry/types of chemical reactions/figure1.png}
            \end{figure}
       }

    And we can see that the cations and the anions switched (alignment is messed up). 
    \item{And for this example, ``the image speaks for itself'' - Hans Niemann
            \begin{figure}[htb!]
            \centering
                \includegraphics[scale=0.23]{figures/chemistry/types of chemical reactions/figure2.png}
            \end{figure}
        }
    
\end{enumerate}

\subsubsection{Problems}
\begin{enumerate}
\setlength\itemsep{0.5em}
    \item{Predict the product of 
            \[
                \ce{Pb(NO3)2_{(aq)}}+ \ce{KI_{(aq)}}\to\quad?
           \]
        }
        \textit{\textbf{Solution:}} We will be swapping lead and potassium since they are the cations 
        \[
                \ce{Pb(NO3)2_{(aq)}}+ \ce{KI_{(aq)}}\to \ce{KNO3_{(aq)}}+ \ce{PbI2_{(s)}}
        \]
        Lead(II) nitrate and potassium iodide are colourless solutions. They react to form potassium nitrate (also colourless solution) and lead(II) iodide. Lead(II) iodide precipitates out of solution and is a yellow solid; so it is very obvious when the reaction happens.

        \item{Predict the products of the following
                \begin{enumerate}[label=(\alph*)]
                    \item{$ \ce{CuS}+ \ce{KCl}\to\quad?$}
                    \item{$ \ce{CaBr2}+ \ce{KOH}\to\quad?$}
                    \item{$ \ce{Ca(OH)2}+ \ce{H3PO4}\to\quad?$}
                \end{enumerate}
            }

        \textit{\textbf{Solutions:}} 
        \begin{enumerate}[label=(\alph*)]
            \item{Copper is multivalent, but it has a higher tendency to form as a $2+$ ion. Then, copper and potassium will switch
        \begin{align*}
            \ce{CuS}+ \ce{KCl}&\to \text{Cu}^{2+} \text{Cl}^{-}+ \text{K}^{+} \text{S}^{2-}\\
            \ce{CuS}+ \ce{KCl}&\to \ce{CuCl2}+ \ce{K2S} 
        \end{align*}}
        \item{Calcium and potassium are metals/cations. Therefore, they will swap
                \begin{align*}
                    \ce{CaBr2}+ \ce{KOH}&\to \text{K}^{+} \text{Br}^{-}+ \text{Ca}^{2+} \text{OH}^{-}\\
                    \ce{CaBr2}+ \ce{KOH}&\to \ce{KBr}+ \ce{Ca(OH)2}
                \end{align*}}
        \item{This kind of double displacement reaction is actually known as a \textbf{Neutralization Reaction}. This type of double displacement reaction only occurs when the product contains water and some kind of salt. In this case, will will swap calcium with hydrogen, since hydrogen is a cation and calcium is also a cation
                \begin{align*}
                    \ce{Ca(OH)2}+ \ce{H3PO4}&\to \text{H}^{-} \text{(OH)}^{-}+ \text{Ca}^{2+} \text{PO}^{3-}_{4}\\
                    \ce{Ca(OH)2}+ \ce{H3PO4}&\to \ce{HOH}+ \ce{Ca3(PO4)2}\\
                    \ce{Ca(OH)2}+ \ce{H3PO4}&\to \ce{H2O}+ \ce{Ca3(PO4)2}\\
                \end{align*}
                Where $ \ce{H2O}$ is water and $ \ce{Ca3(PO4)2}$ is a kind of salt. 
            }
        \end{enumerate}
                
\end{enumerate}

\subsection{Chemical Reactions Problems}
\begin{enumerate}
\setlength\itemsep{1em}
    \item{When lithium metal and oxygen gas react, solid lithium oxide is produced. Identify the type of chemical reaction, and then write the balanced chemical reaction.}\\

        \textit{\textbf{Solution:}} This is a synthesis reaction. The balanced chemical equation will be 
        \begin{align*}
            \ce{Li}+ \ce{O}&\to \text{Li}^{+} \text{O}^{2-}\\
            \ce{Li}+ \ce{O}&\to \ce{Li2O}\\
            \ce{2Li}+ \ce{O}&\to \ce{Li2O}
        \end{align*}

    \item{When a log burns in a fireplace, as shown in the photograph below, the hydrocarbons in the log combine with oxygen gas in the air, producing ashes and gases.
            \begin{enumerate}[label=(\alph*)]
                \item{What two gases are produced?}
                \item{The ashes weigh much less than the log that was burned. Does that mean that this combustion reaction does not follow the law of conservation of mass? Explain.}
            \end{enumerate}
        }

        \textit{\textbf{Solutions:}} 
        \begin{enumerate}[label=(\alph*)]
            \item{In a combustion reaction, according to the general formula, the product is $ \ce{H2O}$ and $ \ce{CO2}$. Therefore, the two gases that will be produced will be carbon gas and oxygen gas (I think).}
            \item{The ashes weighing less than the log doesn't mean that this combustion reactio does not follow the law of conservation of mass. The reason for this is because as per the general chemical formula for combustion reaction, there is oxygen and water produced. Therefore, the mass is not lost, and is instead preserved into those forms.}
        \end{enumerate}

        \item{When aqeuous chlorine reacts with aqeuous potassium bromide, aqeuous potassium chloride and liquid bromine are produced. Identify the type of chemical reaction, and then write the balanced chemical equation, including states.}\\

            \textit{\textbf{Solution:}} This is a single displacement reaction. The balanced chemical equation will be 
            \begin{align*}
                \ce{Cl_{(aq)}}+ \text{K}^{+} \text{Br}^{-}_{\text{(aq)}}&\to \text{K}^{+} \text{Cl}^{-}_{\text{(aq)}}+ \ce{Br^{-}_{( \ell )}}\\
                \ce{Cl_{(aq)}}+ \ce{KBr_{(aq)}}&\to \ce{KCl_{(aq)}}+ \ce{Br_{( \ell )}}\\
            \end{align*}
            And everything is already balanced. 
\end{enumerate}

\newpage
\section{Acids and Bases}
\subsection{Arrhenius Acids and Bases}

\begin{definition}{Arrhenius Acid}
    (1) An arrhenius acid is any species that increases the concentration of $\text{H}^{+}$ in aqeuous solution.\\
    \[
        \ce{HCl_{(aq)}}\to \text{H}^{+}_{\text{(aq)}}+ \text{Cl}^{-}_{\text{(aq)}}
    \]
    We can see that it decomposed into a $\text{H}^{+}$ proton.\\

    (2) A substance that produces $\text{H}^{+}$ when dissolved in water. 
    \[
    \ce{HCl}+ \ce{H2O}\to \ce{H3} \text{O}^{+}+ \ce{Cl}^{-}
    \]
    We can see that it decomposed into a $\ce{H3} \text{O}^{+}$ compound (see note below). 

\end{definition}

\begin{remark}{ } 
    Definition (1) and (2) are both valid; in order to arrive at (2), you must do (1), and thus both are valid. 
\end{remark}

\begin{remark}{ } 
    Using the first definition of an arrhenius acid, we create the postulate that every compound that starts first with hydrogen $ \ce{H}$ is an arrhenius acid. This is because it can always be decomposed into an $\text{H}^{+}_{\text{(aq)}}$ proton. If the compound were to end with hydrogen, then it is not arrhenius, since it will be decomposed into a $\text{H}^{-}_{\text{(aq)}}$ electron. 
\end{remark}

\begin{definition}{Arrhenius Base}
    (1) An arrhenius base is any species that increases the concentration of $\text{OH}^{-}$ in aqeuous solution. 
    \[
        \ce{NaOH_{(aq)}}\to \ce{Na^{+}_{(aq)}}+ \text{OH}^{-}_{\text{(aq)}}
    \]

    (2) A substance that produces $\text{OH}^{-}$ when dissolved in water. 
    \[
        \ce{CH3NH2_{(aq)}}+ \ce{H2O_{(\l)}}\leftrightarrow \ce{CH3} \ce{NH3^{+}_{(aq)}}+ \text{OH}^{-}_{\text{(aq)}}
    \]
\end{definition}
\begin{note}{ }
    Regarding arrhenius acids/bases
    \begin{itemize}
        \item{In aqeuous solutions, $\text{H}^{+}$ ions immediately react with water molecules to form hydronium ions, $\ce{H3} \text{O}^{+}$. The reaction can be written as follows
                \[
                    \text{H}^{+}_{\text{(aq)}}+ \ce{H2O_{( \ell )}}\to \ce{H3} \text{O}^{+}_{\text{(aq)}}
                \]
            Where the hydrogen (proton) is combining with the water to form hydronium.
            }
        \item{In an acid-base or neutralization reaction, an arrhenius acid and base usually react to form water and salt:}
            \[
                \text{arrhenius acid}+ \text{arrhenius base}= \text{water}+ \text{salt}
            \]
    \end{itemize}
\end{note}

Carbonate salts can also form basic solutions. Adding water to carbonate gives 
\[
    \text{CO}^{2-}_{3}+ \ce{H2O}\to \text{HCO}^{-}_{3}+ \text{OH}^{-}
\]
There is an increase in hydroxide $ \text{OH}^{-}$ so it forms a basic solution. If we had say hydrogen carbonate instead of pure carbonate ion instead 
\[
    \text{HCO}^{-}_{3}+ \ce{H2O}\to \ce{H2CO3}
\]
And if it is hydrogen carbonate instead 
\[
    \text{HCO}^{-}_{3}+ \ce{H2O}\to \ce{H2CO3}+ \text{OH}^{-}
\]

\begin{table}[h!] % delete [h!] if there are bugs

    %%% TABLE CONFIG %%% 
    \renewcommand{\arraystretch}{1.5} % changes vertical space for each cell 
    \setlength{\tabcolsep}{10pt} % changes horizontal space for each cell
    \setlength{\arrayrulewidth}{0.25mm}

    \vspace{0.5em}
    \begin{center}
        Properties of Acids and Bases \\
        \vspace{0.5em}
        \begin{tabular}{|l|l|} % use r, l, c for right, left, center. use m{3cm} for middle width of 3cm, use  b{3cm} for bottom width of 3cm, and use p{3cm} for a top width of 3cm.  
        \hline
        Acids & Bases \\ % two columns corresponding to two c's
        \hline
        - Conducts electricity & - Conducts electricity \\ % second row
        - Tastes sour & - Tastes bitter\\ 
        - Neutralizes bases & - Neutralizes acids\\ 
        \hline
        \multicolumn{2}{|c|}{Commonly Found in}\\  
        \hline
        - Preservatives & - Cleaning agents\\
        - Stomache & - Antacids\\ 
        - Citruis fruits & - Baking soda\\ 
        - Soda & \\ 
        \hline
        \end{tabular}
    \end{center}
    \caption{}
\end{table}

\begin{figure}[H]
\centering
    \includegraphics[scale=0.35]{figures/chemistry/litmus-test.png}
    \caption{litmus test's are used to determine acidity}
\end{figure}

\subsection{Naming Acids and Bases}
The naming of acids are broken into two different kinds: \textbf{Common Binary Acid} (element anion) and (polyatomic anion) \textbf{Common Oxyacid}.
\begin{note}{Common Binary Acid}
    These acids all begin with the word ``hydro'', followed by the anion's name ending with ``ic''. Additionally, all of these acids are aqeuous.
\end{note}

\begin{table}[h!] % delete [h!] if there are bugs
        
            %%% TABLE CONFIG %%% 
    \renewcommand{\arraystretch}{1.5} % changes vertical space for each cell 
    \setlength{\tabcolsep}{10pt} % changes horizontal space for each cell
    \setlength{\arrayrulewidth}{0.25mm}

    \begin{center}
        Common Binary Acids \\
        \vspace{0.5em}
        \begin{tabular}{|l|l|} % use r, l, c for right, left, center. use m{3cm} for middle width of 3cm, use  b{3cm} for bottom width of 3cm, and use p{3cm} for a top width of 3cm.  
        \hline
        Chemical Formula & Acid Name \\ % two columns corresponding to two c's
        \hline
        $\ce{HF_{(aq)}}$ & Hydrofluoric acid \\ % second row
        \hline
        $\ce{HCl_{(aq)}}$ & Hydrochloric acid\\ 
        \hline
        $\ce{HBr_{(aq)}}$ & Hydrobromic acid\\
        \hline
        $\ce{HI_{(aq)}}$ & Hydroiodidic acid\\
        \hline 
        $\ce{H2S_{(aq)}}$ & Hydrosulfuric acid\\ 
        \hline
        \end{tabular}
    \end{center}
\end{table}

\begin{table}[h!] % delete [h!] if there are bugs

    %%% TABLE CONFIG %%% 
    \renewcommand{\arraystretch}{1.5} % changes vertical space for each cell 
    \setlength{\tabcolsep}{10pt} % changes horizontal space for each cell
    \setlength{\arrayrulewidth}{0.25mm}

    \begin{center}
        Common Oxyacids \\
        \vspace{0.5em}
        \begin{tabular}{|l|l|} % use r, l, c for right, left, center. use m{3cm} for middle width of 3cm, use  b{3cm} for bottom width of 3cm, and use p{3cm} for a top width of 3cm.  
        \hline
        Chemical Formula & Acid Name \\ % two columns corresponding to two c's
        \hline
        $\ce{CH3COOH_{(aq)}}$ & Acetic acid \\ % second row
        \hline
        $\ce{H2CO3_{(aq)}}$ & Carbonic acid\\
        \hline 
        $\ce{HNO3_{(aq)}}$ & Nitric acid\\ 
        \hline 
        $\ce{H3PO4_{(aq)}}$ & Phosphoric acid\\ 
        \hline 
        $\ce{H2SO4_{(aq)}}$ & Sulfuric acid\\ 
        \hline
        \end{tabular}
    \end{center}
\end{table}

\newpage
To name a base, simply write the first elements name and then write ``hydroxide'' after it.

\begin{table}[h!] % delete [h!] if there are bugs

    %%% TABLE CONFIG %%% 
    \renewcommand{\arraystretch}{1.5} % changes vertical space for each cell 
    \setlength{\tabcolsep}{10pt} % changes horizontal space for each cell
    \setlength{\arrayrulewidth}{0.25mm}

    \begin{center}
        Common Bases \\
        \vspace{0.5em}
        \begin{tabular}{|c|c|c|} % use r, l, c for right, left, center. use m{3cm} for middle width of 3cm, use  b{3cm} for bottom width of 3cm, and use p{3cm} for a top width of 3cm.  
        \hline
        Chemical Formula & Name & Alternate Name\\ % two columns corresponding to two c's
        \hline
        $\ce{NaOH_{(aq)}}$ & Sodium hydroxide & Lye or caustic soda \\ % second row
        \hline 
        $\ce{LiOH_{(aq)}}$ & Lithium hydroxide & \\ 
        \hline 
        $\ce{KOH_{(aq)}}$ & Potassium hydroxide & Caustic potash\\ 
        \hline 
        $\ce{Ca(OH)}$ & Calcium hydroxide & Slaked lime\\ 
        \hline
        $\ce{Mg(OH)2}$ & Magnesium hydroxide & Milk of mangesia\\ 
        \hline 
        $\ce{Al(OH)3}$ & Aluminum hydroxide & \\
        \hline 
        $\ce{NaHCO3}$ & Sodium hydrogen carbonate & Baking soda\\
        \hline
        \end{tabular}
    \end{center}
\end{table}

\begin{+problems}
\item{Name the following and categorize each as an acid or a base:}
    \begin{enumerate}[label=(\alph*)]
     \item{ $\ce{H3PO4_{(aq)}}$}
        \begin{solution}
            We write $\ce{H3PO4_{(aq)}}= \text{H}^{+} \ce{PO4^{3-}_{(aq)}}$. This decomposes into $\text{H}^{+}$ and $\text{PO}^{3-}_{4}$, and there will be a concentration increase of $\text{H}^{+}$. Therefore, this is an arrhenius acid. 
        \end{solution}

    \item{ $\ce{H2S_{(aq)}}$}
        \begin{solution}
            This is an arrhenius acid.
        \end{solution}
    \item{ $\ce{NH4OH_{(aq)}}$}
        \begin{solution}
            This decomposes into $\text{NH}^{+}_{4}$ and $\text{OH}^{-}$, so there is an increase in $\text{OH}^{-}$ and therefore this is an arrhenius base. 
        \end{solution}
    \item{ $\ce{Ca(HCO3)2}$}
        \begin{solution}
            This is carbon hydrogen carbonate, which is a special case (I think) of an arrhenius base. 
        \end{solution}
    \end{enumerate}
\end{+problems}

\subsection{Neutralization Reaction}
\begin{definition}{Neutralization Reaction}
A neutralization reaction can be defined as a chemical reaction in which an \textit{acid} and \textit{base} quantitatively react together to form a salt and water compound. In a neutralization, there is a combination of $ \text{H}^{+}$ ions and $ \text{OH}^{-}$ ions which form water. The general formula is 
\begin{center}
    \framebox{$ \ce{HX}+ \ce{YOH}\to \ce{YX}+ \ce{H2O}$}
\end{center}
Where we use $X$ and $Y$ instead of $A$ and $B$. An example of a neutralization reaction is demonstrated below 
\end{definition}

\begin{figure}[htb!]
\centering
    \includegraphics[scale=0.15]{figures/chemistry/types of chemical reactions/neutralization-reaction.png}
\end{figure}

\begin{remark}{ } 
    The salts formed may be soluble or insoluble. If a salt is insoluble, a precipitate will form. 
\end{remark}

\begin{2example}{ }
    \invis
\begin{list0.5}
    \item{Sulfuric acid mixes with sodium hydroxide (base)
            \[
                \ce{H2SO4_{(aq)}}+ \ce{2NaOH_{(s)}}\to \ce{Na2SO4_{(aq)}}+ \ce{2H2O_{( \ell )}}
            \]
        }
\end{list0.5} 
\end{2example}

\begin{+problems}
    \item{Predict and balance the chemical equation for $ \text{Hydrobromic Acid}+ \text{Sodium Hydroxide}$.}
        \begin{solution}
            $ \text{Hydrobromic Acid}= \ce{HBr_{(aq)}}$ and $ \text{Sodium Hydroxide}= \ce{NaOH_{(aq)}}$. Since this is a neutralization reaction, the balanced chemcial equation is 
            \[
                \ce{HBr_{(aq)}}+ \ce{NaOH_{(aq)}}\to \ce{NaBr{(s)}}+ \ce{H2O_{( \ell )}}
            \]
        \end{solution}

    \item{Predict and balance the chemical eqauation for $\ce{HI_{(aq)}}+ \ce{NH4OH_{(aq)}}$.}
        \begin{solution} 
            Swap iodine with hydroxide so that you form a salt and water 
            \[
                \ce{HI_{(aq)}}+ \ce{NH4OH_{(aq)}}\to \ce{NH4I_{(s)}}+ \ce{H2O_{( \ell )}} 
            \]
        \end{solution}

    \item{Predict and balance the chemical equation for $ \text{Phosphoric Acid}+ \text{Barium Hydroxide}$.}
        \begin{solution}
            $ \text{Phosphoric Acid}= \ce{H3PO4_{(aq)}}$ and $ \text{Barium Hydroxide}= \ce{Ba(OH)2_{(aq)}}$. The balanced chemical equation is 
            \begin{align*}
                \ce{H3PO4_{(aq)}}+ \ce{Ba(OH)2_{(aq)}}&\to \ce{Ba3(PO4)2_{(s)}}+ \ce{H2O_{( \ell )}}\\ 
                \ce{2H3PO4_{(aq)}}+ \ce{3Ba(OH)2_{(aq)}}&\to \ce{Ba3(PO4)2_{(s)}}+ \ce{3H2O_{( \ell )}}
            \end{align*}
        \end{solution}
    \item{Predict and balance the equation when phosphoric acid is mixed with sodium hydroxide.}
        \begin{solution}
            Phosphoric acid is $\text{H}^{+} \text{PO}^{3-}_{4}= \ce{H3PO4}$ and sodium hydroxide is $\ce{NaOH}$. The balanced chemical equation is 
            \begin{align*}
                \ce{H3PO4}+ \ce{NaOH}&\to \ce{Na3PO4}+ \ce{H2O}\\
                \ce{H3PO4}+ \ce{3NaOH}&\to \ce{Na3PO4}+ \ce{H2O}\\
                \ce{2H3PO4}+ \ce{3NaOH}&\to \ce{2Na3PO4}+ \ce{H2O}\\
                \ce{2H3PO4}+ \ce{6NaOH}&\to \ce{2Na3PO4}+ \ce{H2O}\\ 
                \ce{2H3PO4}+ \ce{6NaOH}&\to \ce{2Na3PO4}+ \ce{6H2O}
            \end{align*}
        \end{solution}
\end{+problems}

\newpage
\section{pH - Potential Hydrogen}
\begin{definition}{Potential Hydrogen pH}
   Potential hydrogen, or Power of Hydrogen (pH), is a scale used to specify the acidity or basicity of an aqeuous solution. The lower the pH, the more acidic.   
\end{definition}

\begin{note}{pH scale and range}
    The pH scale is logarithmic. \textbf{pH of 7 is neutral.}  
\end{note}

The change in pH is logarithmic, so for every $n$ difference in pH value, the difference between the acidity is $10^n$. 

\begin{example}{Substance A has a pH of 13.2. Substance B has a pH of 8.5.
        \begin{2qu}
            \item{Which substance is more acidic?}
            \item{Which solution is more basic?}
            \item{How many times did the concentration change from substance A to substance B?}
        \end{2qu}
    }
    \invis
    \begin{2qu}
        \item{Substance B}
        \item{Substance A}
        \item{$10^{13.2-8.5}\approx 50\,119$ times more}
    \end{2qu}
\end{example}

\subsection{Phenolphthalein}
\begin{definition}{Phenolpthalein}
    Phenolphthalein is a pH indicator with the chemical formula $\ce{C20H14O4}$. It is colourless in acidic solutions and pink in basic solutions. 
\end{definition}

\begin{figure}[H]
\centering
    \includegraphics[scale=0.2]{figures/ph-indicator.jpg}
    \caption{Very acidic, acidic, little above basic, above basic}
\end{figure}

\subsection{pH in Agriculture}
Plants are sensitive to the pH of soil. 
\begin{list0.5}
    \item{Legumes grows well in slightly basic pH (7-10)}
    \item{Corn grows well in mildly acidic soil (5-6)}
    \item{Potatoes grow well in acidic soil (pH < 5)}
\end{list0.5}

\subsection{Acid Leaching}
\begin{definition}{Acid Leaching}
    Acid leaching is the process of adding acids to soil that has been highly contaminated with base metals. Phytoremediation is the process of using plants to absorb metal toxins. 
\end{definition}

\begin{2example}{ }
\invis
\begin{list0.5}
    \item{Sunflowers absorb toxic metals and radioactive elements.}
    \item{Aspen trees remove lead from water.}
\end{list0.5}
\end{2example}

\subsection{pH in Consumer Products}
Cleaning products can either be very acidic or basic
\begin{table}[h!] % delete [h!] if there are bugs

    %%% TABLE CONFIG %%% 
    \renewcommand{\arraystretch}{1.5} % changes vertical space for each cell 
    \setlength{\tabcolsep}{10pt} % changes horizontal space for each cell
    \setlength{\arrayrulewidth}{0.25mm}

    \begin{center}
        pH in consumer products \\
        \vspace{0.5em}
        \begin{tabular}{|c|c|} % use r, l, c for right, left, center. use m{3cm} for middle width of 3cm, use  b{3cm} for bottom width of 3cm, and use p{3cm} for a top width of 3cm.  
        \hline
        Product & pH \\ % two columns corresponding to two c's
        \hline
        Bleach & 11-13 \\ % second row
        \hline
        Oven cleaner & 11-13\\
        \hline
        Windex & 9\\
        \hline 
        Dish soap & 7-8\\
        \hline 
        Toilet bowl cleaner & 1-3\\
        \hline 
        Battery acid & 0-1\\
        \hline
        \end{tabular}
    \end{center}
\end{table} 

\subsection{pH in Swimming Pools}
\begin{itemize}
    \item{pH of swimming pools should range from 7.2-7.8}
    \item{7.4 pH is the sweet spot of swimming pools}
    \item{If pH < 7, pool water will irritate eyes}
    \item{If pH > 8, pool water becomes cloudy and chlorine compound will lose its disenfecting ability}
\end{itemize}

Pool pH test kits are used to monitor pH 
\begin{itemize}
    \item{Add $\ce{HCl_{(aq)}}$ (muriatic acid) to reduce pool pH}
    \item{Add $\ce{Na2CO2_{(aq)}}$ to increase pool pH}
\end{itemize}

\subsection{Acid Deposition}
\begin{definition}{Acid Deposition}
    The mix of air pollutants and water that leads to acidification of soil and water systems. 
\end{definition}

\begin{remark}{ } 
    Rainwater is naturally acidic, with a pH of 5.6, due to a prescence of dissolved $\ce{CO2}$ to form $\ce{H2CO3_{(aq)}}.$
\end{remark}

\end{document}
